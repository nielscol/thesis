\documentclass[twoside,10pt,a4paper]{article}
% \usepackage[latin1]{inputenc}
\usepackage[utf8]{inputenc} % usually not needed (loaded by default)
\usepackage[T1]{fontenc}
% \usepackage{amsmath}
\usepackage{amsfonts}
% \usepackage{amssymb}
\usepackage{amsmath,amsthm,amssymb}
\usepackage{graphicx}
\usepackage{hyperref}
% \usepackage{courier}
\usepackage{placeins}
\usepackage{fancyhdr}
\usepackage{color}
% \usepackage{xcolor}
\usepackage{listings}
\usepackage{geometry}
\usepackage{tabularx}
\usepackage[table]{colortbl}
\usepackage{placeins}
% \usepackage{cite}
\usepackage{subcaption}
\usepackage{lipsum}
\usepackage{titlesec}
\usepackage{dsfont}
\usepackage{parskip}
\usepackage{ragged2e}
\usepackage{soul}
\usepackage{tablefootnote}
\usepackage{epigraph}
\usepackage{setspace}
\usepackage[backend=biber, sorting=none]{biblatex}
\addbibresource{mybib.bib}
\usepackage[skip=8pt,labelfont=bf, font=sf]{caption}
	\DeclareCaptionFont{myfont}{\fontfamily{\sfdefault}\selectfont}

\hypersetup{
	colorlinks   = True,
	citecolor    = blue,
	linkcolor	 = blue,
}

% \geometry{letterpaper, portrait, margin=1in}
% \geometry{a4paper,margin=15mm,bindingoffset=25mm,heightrounded,}
\geometry{a4paper,margin=25mm}

\usepackage[dvipsnames]{xcolor}
\definecolor{lightgray}{rgb}{.93,.93,.93}
\definecolor{darkgray}{rgb}{.4,.4,.4}
\definecolor{purple}{rgb}{0.65, 0.12, 0.82}

\usepackage{array}
\newcolumntype{?}{!{\vrule width 1.5pt}}

\usepackage{helvet}
\usepackage{times}

\titleformat{\chapter}[display]
  {\normalfont\sffamily\huge\bfseries} % \color{blue}
  {\chaptertitlename\ \thechapter}{20pt}{\Huge}
\titleformat{\section}
  {\normalfont\sffamily\huge\bfseries} % \color{cyan}
  {\thesection}{1em}{}
 \titleformat{\subsection}
  {\normalfont\sffamily\Large\bfseries} % \color{cyan}
  {\thesubsection}{1em}{}
 \titleformat{\subsubsection}
  {\normalfont\sffamily\large\bfseries} % \color{cyan}
  {\thesubsubsection}{1em}{}
\renewenvironment{abstract}
 {\par\noindent\Huge\textbf{\abstractname.}\\\vspace{1em}\ignorespaces}
 {\par\medskip} 

% \renewcommand*\abstractname{Abstract\hfill}
% \renewcommand{\familydefault}{\sfdefault}


\def \lineheight {1.5pt}

\makeatletter
   \def\vhrulefill#1{\leavevmode\leaders\hrule\@height#1\hfill \kern\z@}
\makeatother

\usepackage{caption}
    \DeclareCaptionType{mycapequ}[][List of equations]
    \captionsetup[mycapequ]{labelformat=empty}

\usepackage{inconsolata}

\lstset{
	language=Matlab,
	backgroundcolor=\color{lightgray},
	keywordstyle=\color{blue}\bfseries,
	stringstyle=\color{red}\ttfamily,
	commentstyle=\color{ForestGreen}\ttfamily,
	identifierstyle=\color{black},
	extendedchars=true,
	% basicstyle=\footnotesize\ttfamily,
	basicstyle=\normalsize\fontencoding{T1}\ttfamily,
	showstringspaces=false,
	showspaces=false,
	numbers=left,
	numberstyle=\footnotesize,
	numbersep=9pt,
	tabsize=2,
	breaklines=true,
	showtabs=false,
	captionpos=b
}

\lstset{
	language=Python,
	backgroundcolor=\color{lightgray},
	keywordstyle=\color{blue}\bfseries,
	stringstyle=\color{red}\ttfamily,
	commentstyle=\color{ForestGreen}\ttfamily,
	identifierstyle=\color{black},
	extendedchars=true,
	% basicstyle=\footnotesize\ttfamily,
	basicstyle=\normalsize\fontencoding{T1}\ttfamily,
	showstringspaces=false,
	showspaces=false,
	numbers=left,
	numberstyle=\footnotesize,
	numbersep=9pt,
	tabsize=2,
	breaklines=true,
	showtabs=false,
	captionpos=b
}

\fancypagestyle{firstpage}{%
  \fancyhf{}% clear default for head and foot
  % \rfoot{\textbf{Page \thepage}}
  % \lfoot{\textbf{\today}}
  \renewcommand{\headrulewidth}{0pt}
}
\fancypagestyle{blank}{%
	\fancyhf{}
	\renewcommand{\headrulewidth}{0pt}
}

\fancypagestyle{nohdr}{%
	\fancyhf{}
	\renewcommand{\headrulewidth}{0pt}
	% \rfoot{\fontfamily{\sfdefault}\selectfont\textbf{Page \thepage}}
	\fancyfoot[LE,RO]{\fontfamily{\sfdefault}\selectfont \textbf{Page \thepage}}
}

\renewcommand{\headrulewidth}{1pt}% 2pt header rule


% \fontfamily{\sfdefault}\selectfont
	% \large\fontfamily{\rmdefault}\selectfont 

\setlength{\parskip}{12pt} % 1ex plus 0.5ex minus 0.2ex}
\setlength{\parindent}{0pt}
% \linespread{1.5}
\onehalfspacing

\DeclareMathOperator*{\argmin}{argmin}

\usepackage{tikz}
\usepackage{enumitem}
\newcommand*\mycirc[1]{%
\begin{tikzpicture}[baseline=(C.base)]
\node[draw,circle,inner sep=1pt,minimum size=3ex](C) {#1};
\end{tikzpicture}}