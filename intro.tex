Phase locked loops (PLLs) are the fundamental building block to virtually all wired and wireless communication systems of today. To meet industrial demands of continual and uncompromsing improvement of communication system performance, i.e. higher data rates, lower power, it is paramount that PLL performance is continually improved. The advent of battery powered mobile and IoT produces an accute need for power reduction. A recent approach to reducing power consumption of mobile and IoT devices is through usage of wake up receivers (WUR). These are ultra low power, low data rate radio receivers, which listen for requests (i.e. a "wake up call", or WUC) for activity of the aforementioned devices. Upon a WUC, the device activates a higher powered radio supporting higher data rates for only the time required. In devices which are inactive for large periods of time, waiting for requests for activity (e.g. as sensor networks or wireless headphones), such a scheme can enable great power reduction, achieving 4.5 nW in \cite{Jiang2017} and 365 nW in \cite{Sadagopan2017} for 2.4 GHz reception, compared to utilizing a full data rate receiver to poll the radio spectrum for activity requests. 

Thus, in this work, low power PLL design which enables WUR design is to be considered. Ultra low power has been achieved with PLL-less OOK receivers, for example achieving 4.5 nW with 0.3 kbps of data at 2.4GHz \cite{Jiang2017}. However, this work will be catered to PLL-based designs that mantain backwards-compatibility with FSK, PSK modulation schemes supported by existing standards such as 802.15.4, Wifi, Bluetooth. The PLL design approached in this work will seek methods to reduce overall complexity (minimize current paths), whilst yielding high performance on a given power budget. A brief outline of the paper is as follows. An introduction to PLL and FD-SOI theory is in section \ref{theory}. The undertaken PLL Design are discussed in section \ref{design}. Simulation results obtained of the design are in section \ref{results}. Comparisons to states of art and general discussion regarding this work is in section \ref{disco}. Finally, section \ref{conclusion} concludes. 
%
%
% Phase locked loops are extraordinarily useful frequency synthesizers that are vital to the operation of virtually all wThe trend towards increasingly lower power wireless devices poses an acute need to reduce PLL power consumption. This is a challenge as PLLs typically rank among the highest power consuming components of a radio, and are necessarily so to limit oscillator phase noise. A sampling of literature on ultra-low power 2.4GHz radios finds oscillator power consumption as a portion of total radio consumption to be 53\% for the receiver in \cite{regulagadda_2018}, 88\% of the transmitter in \cite{shi_2019}, 52\% of the transmitter in \cite{chen_2019}, and 50\% of the receiver in \cite{pengg_2013}. Reducing analog PLL power consumption can be a prohibitive challenge as the performance of analog loop filters degrade as a result of unavoidably lower charge pump current. However, recent CMOS process nodes with minimum gate lengths as small as 7nm allow for all-digital loop filters and PLLs to be a possible alternative to analog designs due to increasingly low power consumption associated with their implementation. Digital loop-filters have the unique advantage where they can be scaled indefinitely as process nodes advance, suffering no loss in performance, while also having greatly reduced sensitivities to process, voltage, temperature (PVT) variations compared to analog implementations.
%
%
% Thus, in this paper, a new framework, \texttt{pllsim}, written in Python\footnote{Python Software Foundation \url{https://www.python.org/}.} is introduced (this framework is available on GitHub\footnote{\texttt{pllsim} codebase: \url{https://github.com/nielscol/pllsim}.}), which uniquely addresses issues of ultra-low power ADPLL design. Specifically, design of integer-N type PLLs is focused on, as the impetus of this work is an integer-N PLL design project. Topics presented are (a) automatic design and optimization of ADPLL loop filters given target system and component level specifications for the PLL, and (b) behavioral time domain PLL simulation for accurate analysis and verification of loop filter and PLL performance, with an integrated Monte-Carlo sampling variation analysis engine. Due to high phase noise associated with low power design, the optimization approach introduced in this paper focuses on the minimization of total integrated phase noise power to allow for maximum PLL performance on a given power budget.
%

\vspace{1em}

\subsection{Main Contributions}
\vspace{-0.8em}
\begin{enumerate}[itemsep=0pt,label=\protect\mycirc{\arabic*}]
	\setlength\itemsep{-0.8em}
	\item Implementation of an ultra-low power, 0.0051 mm$^2$ area CMOS PLL in 22FDX FD-SOI technology.
	\item Presentiation of a novel pseudodifferential ring oscillator circuit topology and operation theory, utilizing FD-SOI backgates to implement both frequency tuning and differential coupling.
	\item Realization of linear gain voltage controlled oscillator with rail to rail range. 
	\item Loop filter optimization theory for bang-bang phase dector PLL with a noisy detector.
	\item DAC/oscillator gain optimization theory.
	\item Novel pseudodifferential buffer presenting common mode rejection characteristics.
	\item Implementation of low power CDACs, bang-bang phase detector.
	\item Implementation of low power digital loop filter.
	\item Demonstration of bias current and reference free PLL design.
\end{enumerate}
