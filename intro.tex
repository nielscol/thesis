Phase locked loops (PLLs) are the fundamental building block to virtually all wired and wireless communication systems of today. To meet industrial demands of continual and uncompromising improvement of communication system performance, i.e. higher data rates, lower power, it is paramount that PLL performance is continually improved. Of perpetually growing importance is the application of PLLs to radios in battery powered mobile and IoT devices, for which reduction of power is highly sought after. A recent approach to reducing power consumption in such wireless applications is through usage of wake up receivers (WUR). These are ultra low power, low data rate radio receivers, which listen for requests (i.e. a "wake up call", or WUC) for activity from some external source. Upon a WUC, the device powers on and activates a higher powered radio supporting higher data rates for only the time required. In devices which are inactive for substantial periods of time, waiting for requests for activity (e.g. as with sensor networks or wireless headphones), such a scheme can enable great power reduction, for example achieving 4.5 nW in \cite{Jiang2017} and 365 nW in \cite{Sadagopan2017} for 2.4 GHz band WUC reception. When this is compared to utilizing a full data rate receiver to poll the radio spectrum for activity requests, which for a state of art Bluetooth design may draw on the order of 1.9 mW \cite{Tamura2020}, it is seen that upwards of $10^6$ improvement in power is obtainable, undoubtedly reducing power consumption.

Thus, in this work, the design of a low power PLL which enables WUR applications is considered. Ultra low power consumption has been achieved with PLL-less on-off keying receivers, for example achieving 4.5 nW with 0.3 kbps of data at 2.4 GHz \cite{Jiang2017}. However, this work will be catered to PLL-based designs that maintain backwards-compatibility with FSK and PSK modulation schemes supported by existing wireless standards such as 802.15.4, WiFi and Bluetooth. A review of current literature shows that state of art within ultra low power PLLs in the 2.4GHz band regime achieve power consumption on the order of hundreds of $\mu$W, for example 170 $\mu$W in \cite{Zhang2019}, 265 $\mu$W in \cite{Liu2019}. Therefore, to advance the boundary of current state of art, this work seeks to achieve a new record for PLL power consumption, namely $\leq$ 100 $\mu$W for use in 2.4 GHz band radio operation. Furthermore, an attempt will be made to minimize implemented area of the PLL. Current state of art for PLL area rests in the sub-0.01 mm$^2$ regime, with as small as 0.0036 mm$^2$ being achieved in 5nm process technology \cite{Liu2020}. It will be attempted to obtain a similar area to the current state of art.
% To meet these goals, a PLL design methodology is described in this work, favoring minimization of overall complexity,educing current braches and circuit area, whilst yielding high performance on a given power budget.


 A brief outline of the paper is as follows. An introduction to PLL and FD-SOI theory is in section \ref{theory}. The undertaken PLL Design is discussed in sections \ref{pll_arch}-\ref{behav_sim}. Simulation results of the implemented design are in section \ref{results}. Comparison to the state of art and general discussion regarding this work is in section \ref{disco}. Finally, section \ref{conclusion} concludes. 
%
%
% Phase locked loops are extraordinarily useful frequency synthesizers that are vital to the operation of virtually all wThe trend towards increasingly lower power wireless devices poses an acute need to reduce PLL power consumption. This is a challenge as PLLs typically rank among the highest power consuming components of a radio, and are necessarily so to limit oscillator phase noise. A sampling of literature on ultra-low power 2.4GHz radios finds oscillator power consumption as a portion of total radio consumption to be 53\% for the receiver in \cite{regulagadda_2018}, 88\% of the transmitter in \cite{shi_2019}, 52\% of the transmitter in \cite{chen_2019}, and 50\% of the receiver in \cite{pengg_2013}. Reducing analog PLL power consumption can be a prohibitive challenge as the performance of analog loop filters degrade as a result of unavoidably lower charge pump current. However, recent CMOS process nodes with minimum gate lengths as small as 7nm allow for all-digital loop filters and PLLs to be a possible alternative to analog designs due to increasingly low power consumption associated with their implementation. Digital loop-filters have the unique advantage where they can be scaled indefinitely as process nodes advance, suffering no loss in performance, while also having greatly reduced sensitivities to process, voltage, temperature (PVT) variations compared to analog implementations.
%
%
% Thus, in this paper, a new framework, \texttt{pllsim}, written in Python\footnote{Python Software Foundation \url{https://www.python.org/}.} is introduced (this framework is available on GitHub\footnote{\texttt{pllsim} codebase: \url{https://github.com/nielscol/pllsim}.}), which uniquely addresses issues of ultra-low power ADPLL design. Specifically, design of integer-N type PLLs is focused on, as the impetus of this work is an integer-N PLL design project. Topics presented are (a) automatic design and optimization of ADPLL loop filters given target system and component level specifications for the PLL, and (b) behavioral time domain PLL simulation for accurate analysis and verification of loop filter and PLL performance, with an integrated Monte-Carlo sampling variation analysis engine. Due to high phase noise associated with low power design, the optimization approach introduced in this paper focuses on the minimization of total integrated phase noise power to allow for maximum PLL performance on a given power budget.
%

\vspace{1em}

\subsection{Main Contributions}
% \vspace{-0.8em}
\begin{enumerate}[itemsep=0pt,label=\protect\mycirc{\arabic*}]
	\setlength\itemsep{-0.8em}
	\item Implementation of a sub-100 $\mu$W ultra-low power, 0.00365 mm$^2$ area CMOS PLL in a 22nm FD-SOI process technology, with state of art FOM$_{jitter}$ within its power regime, and comparable area to current state of art.
	\item Presentation of a novel pseudodifferential ring oscillator circuit topology and theory of operation, utilizing FD-SOI backgates to implement both frequency tuning and differential coupling.
	\item Realization of linear gain voltage controlled oscillator with rail to rail input range. 
	\item Loop filter optimization theory for proportional-integral controller bang-bang phase detector PLLs with noisy phase detectors.
	\item Theoretical figure for FOM$_{jitter}$ performance limit of proportional-integral controller bang-bang phase detector PLLs.
	\item DAC resolution and oscillator frequency gain optimization theory.
	\item A novel pseudodifferential buffer presenting common mode rejection characteristics.
	\item Implementation of low power CDACs.
	\item Implementation of a low power bang-bang phase detector.
	\item Implementation of a low power digital loop filter.
	\item Demonstration of bias current and reference free PLL design.
\end{enumerate}
